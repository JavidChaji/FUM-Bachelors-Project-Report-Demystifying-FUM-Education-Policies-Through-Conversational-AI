\thispagestyle{empty}
\chapter*{مقدمه}
\addcontentsline{toc}{chapter}{مقدمه}
{ در عصر حاضر، هوش مصنوعی ( \footnote{Artificial Intelligence}AI ) به سرعت در حال تغییر دنیای ما است. یکی از زمینه‌های نوظهور هوش مصنوعی، استفاده  از آن در آموزش است. چت‌بات‌های آموزشی، ربات‌های گفتگوی مبتنی بر AI هستند که می‌توانند برای ارائه پشتیبانی و راهنمایی آموزشی به دانشجویان استفاده شوند.

چت‌بات‌های آموزشی می‌توانند در پاسخگویی به سؤالات دانشجویان در مورد موضوعات مختلف آموزشی، از جمله قوانین و مقررات، مفید باشند. به عنوان مثال، یک چت‌بات آموزشی می‌تواند به یک دانشجو در مورد قوانین غیبت در کلاس یا قوانین تحصیل در مقطع دکترا پاسخ دهد.

در این پروژه، ما به بررسی امکان ساخت یک چت‌بات آموزشی برای پاسخگویی به سؤالات دانشجویان در مورد قوانین و مقررات آموزشی می‌پردازیم. ما از چارچوب Rasa و پلتفرم Rocket.Chat برای ساخت چت‌بات خود استفاده خواهیم کرد.

بیان مسأله

چالش اصلی در ساخت یک چت‌بات آموزشی برای پاسخگویی به سؤالات دانشجویان در مورد قوانین و مقررات آموزشی، جمع‌آوری مجموعه داده‌های آموزشی کافی است. این مجموعه داده‌ها باید شامل سؤالات متنوعی از دانشجویان در مورد قوانین و مقررات مختلف باشد.

در این پروژه، ما از شیوه نامه آموزشی دانشگاه خود (شیوه‌نامه آموزشی دانشگاه فردوسی مشهد \cite{1}) برای جمع‌آوری مجموعه داده‌های آموزشی استفاده خواهیم کرد. شیوه نامه آموزشی شامل اطلاعات مفیدی در مورد قوانین و مقررات آموزشی است که می‌تواند برای آموزش چت‌بات ما استفاده شود.

اهداف و مقاصد پژوهش

اهداف این پروژه عبارتند از:

ارزیابی مدل‌های زبانی مختلف برای استفاده در چت‌بات آموزشی

ساخت یک چت‌بات آموزشی با استفاده از چارچوب Rasa و پلتفرم Rocket.Chat

اتصال چت‌بات آموزشی به شیوه نامه آموزشی دانشگاه

بیان فرضیه

فرضیه این پروژه این است که می‌توان با استفاده از چارچوب Rasa و پلتفرم Rocket.chat و با جمع‌آوری مجموعه داده‌های آموزشی کافی، چت‌بات آموزشی را برای پاسخگویی به سؤالات دانشجویان در مورد قوانین و مقررات آموزشی ساخت.

نتیجه‌گیری

این مقدمه به طور خلاصه به موضوع پروژه، چالش‌های موجود و اهداف پژوهش می‌پردازد. همچنین، فرضیه پژوهش را بیان می‌کند.

}
{\vspace{1cm}\begin{flushleft}{جاوید چاجی \\ ۱۴۰۲/۰۹/۱۵} \end{flushleft}
\newpage