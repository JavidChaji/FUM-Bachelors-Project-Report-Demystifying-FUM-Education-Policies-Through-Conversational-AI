\chapter{نتیجه‌گیری}

\section{یافته‌های کلیدی و پیامدها}

در این بخش، یافته‌های کلیدی پژوهش و پیامدهای آن‌ها را بررسی خواهیم کرد.

\subsection{یافته‌های کلیدی}

در این مطالعه، یک چت‌بات قوانین آموزشی بر اساس چارچوب Rasa و مدل زبانی SpaCy توسعه یافت و با پلتفرم ارتباطی Rocket.chat ادغام شد. یافته‌های کلیدی این مطالعه عبارتند از:

\begin{itemize}
    \item چت‌بات قوانین آموزشی قادر به پاسخگویی به طیف گسترده‌ای از سوالات آموزشی دانشجویان بود، از جمله سوالاتی در مورد مقررات آموزشی، نحوه ثبت نام در کلاس‌ها، و نحوه درخواست کمک مالی.
    \item چت‌بات قوانین آموزشی با دقت بالایی پاسخ‌های خود را ارائه می‌داد.
    \item ادغام چت‌بات قوانین آموزشی با Rocket.Chat باعث سهولت دسترسی دانشجویان به این چت‌بات می‌شود.
\end{itemize}

\subsection{پیامدها}

یافته‌های این مطالعه نشان می‌دهد که چت‌بات‌های قوانین آموزشی می‌توانند ابزاری ارزشمند برای دانشجویان باشند. این چت‌بات‌ها می‌توانند به دانشجویان در پاسخگویی به سوالات آموزشی خود کمک کنند.

به طور خاص، چت‌بات قوانین آموزشی می‌تواند پیامدهای زیر را داشته باشد:

\begin{itemize}
    \item بهبود تجربه یادگیری دانشجویان: چت‌بات‌های قوانین آموزشی می‌توانند به دانشجویان در پاسخگویی به سوالات خود به سرعت و به راحتی کمک کنند. این امر می‌تواند به دانشجویان در صرفه‌جویی در زمان و انرژی کمک کند و به آنها اجازه دهد تا بر یادگیری خود تمرکز کنند.
    \item کاهش بار کاری کارکنان آموزشی: چت‌بات‌های قوانین آموزشی می‌توانند به کاهش بار کاری کارکنان آموزشی کمک کنند. کارکنان آموزشی می‌توانند از چت‌بات‌های قوانین آموزشی برای پاسخگویی به سوالات تکراری استفاده کنند، که به آنها اجازه می‌دهد تا روی مسائل مهم‌تر تمرکز کنند.
    \item افزایش رضایت دانشجویان: چت‌بات‌های قوانین آموزشی می‌توانند به افزایش رضایت دانشجویان کمک کنند. دانشجویان از اینکه می‌توانند به سوالات خود به سرعت و به راحتی پاسخ دهند، قدردانی می‌کنند.
\end{itemize}


\section{محدودیت‌ها و مسیرهای آینده}

در این بخش، محدودیت‌های چت‌بات قوانین آموزشی و مسیرهای آینده برای بهبود آن بررسی می‌شود.

\subsection{محدودیت‌ها}

یکی از محدودیت‌های اصلی چت‌بات قوانین آموزشی، محدود بودن مجموعه داده‌های آموزشی است. این محدودیت باعث می‌شود که چت‌بات نتواند به تمام سوالات دانشجویان پاسخ دهد. برای رفع این محدودیت، می‌توان مجموعه داده‌های آموزشی را با سوالات جدید و متنوع‌تر گسترش داد.

محدودیت دیگر چت‌بات قوانین آموزشی، عدم توانایی آن در پاسخ به سوالات انتزاعی است. به عنوان مثال، اگر دانشجو از چت‌بات بپرسد که "چه چیزی باعث می‌شود که یک درس دشوار باشد؟"، چت‌بات نمی‌تواند به این سوال پاسخ دهد. برای رفع این محدودیت، می‌توان از مدل‌های زبانی پیچیده‌تر استفاده کرد که توانایی درک و پاسخ به سوالات انتزاعی را دارند.

\subsection{مسیرهای آینده}


برای بهبود چت‌بات قوانین آموزشی، می‌توان مسیرهای زیر را دنبال کرد:

\begin{itemize}
    \item گسترش مجموعه داده‌های آموزشی
    \item استفاده از مدل‌های زبانی پیچیده‌تر
    \item توسعه الگوریتم‌های یادگیری ماشینی جدید برای بهبود عملکرد چت‌بات
\end{itemize}

علاوه بر این، می‌توان چت‌بات قوانین آموزشی را به گونه‌ای توسعه داد که بتواند به سوالات دانشجویان در مورد سایر موضوعات آموزشی نیز پاسخ دهد. به عنوان مثال، چت‌بات می‌تواند به سوالات دانشجویان در مورد منابع آموزشی، تکالیف و امتحانات نیز پاسخ دهد.

در نهایت، می‌توان چت‌بات قوانین آموزشی را به یک پلتفرم تعاملی برای دانشجویان تبدیل کرد. در این پلتفرم، دانشجویان می‌توانند سوالات خود را مطرح کنند، با سایر دانشجویان در مورد موضوعات آموزشی بحث کنند و از منابع آموزشی استفاده کنند.


\section{نتیجه‌گیری}

در مجموع، این پژوهش نشان می‌دهد که چت‌بات قوانین آموزشی می‌تواند ابزاری مفید برای دانشجویان و کارکنان آموزشی باشد. با توسعه بیشتر این چت‌بات، می‌توان آن را به ابزاری قدرتمند برای تسهیل فرایند یادگیری و آموزش تبدیل کرد.

\newpage