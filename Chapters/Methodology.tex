\chapter{روش‌شناسی}

\section{انتخاب مدل زبان}

در این بخش، مدل زبانی مناسب برای توسعه چت‌بات آموزشی مورد بررسی قرار گرفت. مدل‌های زبانی مختلفی برای چت‌بات‌ها وجود دارند که هر یک دارای مزایا و معایب خاص خود هستند. برخی از عوامل مهم در انتخاب مدل زبان عبارتند از:

\begin{itemize}
    \item حجم داده‌های آموزشی: مدل‌های زبانی بزرگ‌تر، به داده‌های آموزشی بیشتری نیاز دارند.
    \item سرعت پردازش: مدل‌های زبانی کوچک‌تر، سریع‌تر پردازش می‌شوند.
    \item دقت پاسخ‌دهی: مدل‌های زبانی دقیق‌تر، پاسخ‌های دقیق‌تری ارائه می‌دهند.
\end{itemize}

در این پروژه، از مدل زبان \lr{SpacyNLP xx\_sent\_ud\_sm} استفاده شد. این مدل زبان، یک مدل زبانی کوچک و سریع است که بر روی مجموعه داده‌های بزرگی از متن و کد آموزش دیده است. این مدل زبان، دقت پاسخ‌دهی مناسبی نیز دارد.

\subsection{دلایل انتخاب مدل \lr{SpacyNLP xx\_sent\_ud\_sm}}

دلایل انتخاب مدل \lr{SpacyNLP xx\_sent\_ud\_sm} عبارتند از:

\begin{itemize}
    \item حجم داده‌های آموزشی: این مدل زبان، بر روی مجموعه داده‌های بزرگی از متن و کد آموزش دیده است که این امر، دقت پاسخ‌دهی آن را افزایش می‌دهد.
    \item سرعت پردازش: این مدل زبان، یک مدل زبانی کوچک و سریع است که این امر، آن را برای توسعه چت‌بات‌های آموزشی مناسب می‌سازد.
    \item چند زبانه بودن: این مدل زبان، بر روی مجموعه داده‌هایی از زبان‌های مختلف آموزش دیده است که این امر، آن را برای توسعه چت‌بات‌های آموزشی چند زبانه مناسب می‌سازد.
    \item آموزش بر روی جملات: این مدل زبان، بر روی مجموعه داده‌هایی از جملات آموزش دیده است که این امر، آن را برای توسعه چت‌بات‌های آموزشی که باید به سوالات و درخواست‌های کاربران پاسخ دهند، مناسب می‌سازد.
\end{itemize}

\subsection{نتیجه‌گیری}

مدل \lr{SpacyNLP xx\_sent\_ud\_sm}،  یک مدل زبانی کوچک، سریع، دقیق و چند زبانه است که برای توسعه چت‌بات‌های آموزشی مناسب می‌باشد. این مدل زبان، بر روی مجموعه داده‌های بزرگی از متن و کد آموزش دیده است که این امر، دقت پاسخ‌دهی آن را افزایش می‌دهد. همچنین، این مدل زبان، سریع است که این امر، آن را برای توسعه چت‌بات‌های آموزشی که باید به سرعت پاسخ‌های کاربران را ارائه دهند، مناسب می‌سازد. علاوه بر این، این مدل زبان، بر روی مجموعه داده‌هایی از زبان‌های مختلف آموزش دیده است که این امر، آن را برای توسعه چت‌بات‌های آموزشی چند زبانه مناسب می‌سازد. همچنین، این مدل زبان، بر روی مجموعه داده‌هایی از جملات آموزش دیده است که این امر، آن را برای توسعه چت‌بات‌های آموزشی که باید به سوالات و درخواست‌های کاربران پاسخ دهند، مناسب می‌سازد.

در این بخش، توضیحاتی در مورد معنای حروف و کلمات استفاده شده در نام مدل زبان \lr{SpacyNLP xx\_sent\_ud\_sm} ارائه شده است.

xx : مخفف چند زبانه بودن (multilingual)

sent : مخفف روی جملات آموزش دیدن (sentences)

ud : مخفف \lr{Universal Dependencies}

sm : مخفف کوچک (small)

بنابراین، نام مدل زبان \lr{SpacyNLP xx\_sent\_ud\_sm}، به معنای مدل زبانی کوچک و سریع چند زبانه است که بر روی مجموعه داده‌هایی از جملات با ساختارهای دستوری جهانی آموزش دیده است.

\subsection{\lr{Universal Dependencies (UD)} چیست؟}
یک چارچوب\footnote{Structure} برای نشان دادن دستور زبان (اجزای گفتار، ویژگی‌های صرفی و روابط دستوری) در زبان‌های مختلف است. این چارچوب بر اساس یک مجموعه از روابط دستوری جهانی است که برای توصیف روابط بین کلمات در هر زبانی استفاده می‌شود.

\lr{Universal Dependencies} \cite{2} در سال 2013 توسط یک گروه از محققان از دانشگاه استنفورد، دانشگاه آکسفورد و دانشگاه کمبریج ایجاد شد. این چارچوب به سرعت مورد پذیرش قرار گرفت و اکنون برای ایجاد بانک‌های درختی برای بیش از 100 زبان استفاده می‌شود.

UD مزایای متعددی دارد. این یک چارچوب انعطاف‌پذیر است که می‌تواند برای توصیف ساختارهای دستوری پیچیده در زبان‌های مختلف استفاده شود. این یک چارچوب باز است که به محققان اجازه می‌دهد روابط دستوری جدیدی را اضافه کنند. و این یک چارچوب رایگان است که برای همه در دسترس است.

UD در طیف گسترده‌ای از برنامه‌های کاربردی استفاده می‌شود. از جمله:

\begin{itemize}
\item پردازش زبان طبیعی (NLP): UD برای توسعه ابزارهای NLP مانند چت‌بات‌ها، ترجمه ماشینی و تشخیص گفتار استفاده می‌شود.
\item آموزش زبان: UD برای توسعه منابع آموزشی مانند فرهنگ لغات و دستور زبان استفاده می‌شود.
\item تحقیقات زبانشناسی: UD برای مطالعه ساختار دستوری زبان‌های مختلف استفاده می‌شود.
\end{itemize}

UD یک چارچوب مهم برای پردازش زبان طبیعی و تحقیقات زبانشناسی است. این چارچوب به محققان این امکان را می‌دهد تا ساختار دستوری زبان‌های مختلف را به روشی استاندارد و سازگار توصیف کنند.


\section{آماده‌سازی داده‌ها و ایجاد مجموعه داده‌های آموزشی}


در این بخش، داده‌های مورد نیاز برای آموزش چت‌بات Rasa از طریق شیوه نامه آموزشی ۱۴۰۰ دانشگاه فردوسی مشهد تهیه شد. این شیوه نامه شامل تعاریف، ماده ها و تبصره های مختلف مربوط به امور آموزشی دانشگاه است. داده‌ها به صورت پرسش و پاسخ تهیه شدند و به صورت زیر دسته‌بندی شدند:

\begin{itemize}
    \item تعاریف: تعاریف مختلف مربوط به امور آموزشی دانشگاه در قالب پرسش و پاسخ تهیه شدند. به عنوان مثال، پرسش "دانشجو چیست؟" با پاسخ "شخصی که در یکی از دانشگاه‌های ایران در حال تحصیل است" همراه شد.
    \item ماده ها: ماده های مختلف شیوه نامه آموزشی به صورت پرسش و پاسخ تهیه شدند. به عنوان مثال، پرسش "ماده 1 شیوه نامه آموزشی چیست؟" با پاسخ "تعریف دانشجو" همراه شد.
    \item تبصره ها: تبصره های مختلف شیوه نامه آموزشی به صورت پرسش و پاسخ تهیه شدند. به عنوان مثال، پرسش "تبصره 1 ماده 1 شیوه نامه آموزشی چیست؟" با پاسخ "شرایط پذیرش دانشجو در دانشگاه" همراه شد.
\end{itemize}

در مجموع، پرسش و پاسخ های زیادی برای آموزش چت‌بات تهیه شد. این پرسش و پاسخ ها به صورت دستی تهیه شدند و سپس با استفاده از Rasa به صورت مجموعه داده‌های آموزشی تبدیل شدند.

\subsection{محدودیت‌های مجموعه داده‌های آموزشی}


مجموعه داده‌های آموزشی تهیه شده دارای برخی محدودیت‌ها است. یکی از محدودیت‌ها این است که این مجموعه داده‌ها فقط شامل تعاریف، ماده ها و تبصره های شیوه نامه آموزشی است. بنابراین، ممکن است پاسخ‌های چت‌بات برای برخی پرسش‌های خارج از این محدوده، دقیق نباشد.

محدودیت دیگر این است که مجموعه داده‌های آموزشی به صورت دستی تهیه شده است. این امر ممکن است باعث شود که برخی پرسش و پاسخ ها ناقص یا دارای اشکال باشند.

\subsection{راهکارهای بهبود مجموعه داده‌های آموزشی}

برای بهبود مجموعه داده‌های آموزشی، می‌توان اقدامات زیر را انجام داد:

مجموعه داده‌ها را با پرسش و پاسخ‌های بیشتری از منابع مختلف، مانند وب سایت‌های دانشگاهی، تکمیل کرد.
از نرم‌افزارهای خودکار برای شناسایی و اصلاح اشکالات موجود در مجموعه داده‌ها استفاده کرد.
با انجام این اقدامات، می‌توان مجموعه داده‌های آموزشی را دقیق‌تر و کامل‌تر کرد و عملکرد چت‌بات را بهبود بخشید.


\section{توسعه چت‌بات Rasa}

پس از آماده‌سازی داده‌های آموزشی، می‌توان چت‌بات را توسعه داد. چت‌بات Rasa از یک چارچوب توسعه چت‌بات متن باز استفاده می‌کند. این چارچوب شامل ابزارها و کتابخانه‌هایی برای توسعه چت‌بات‌های آموزشی است.

در این پروژه، از چارچوب Rasa استفاده شده است. برای توسعه چت‌بات، ابتدا باید یک پروژه Rasa ایجاد شود. سپس، مدل زبان انتخاب‌شده به پروژه اضافه می‌شود. در مرحله بعد، مجموعه داده‌های آموزشی به پروژه اضافه می‌شود.

Rasa از یک فرآیند آموزش خودکار برای آموزش چت‌بات استفاده می‌کند. این فرآیند به چت‌بات اجازه می‌دهد تا از مجموعه داده‌های آموزشی یاد بگیرد و توانایی خود را در پاسخ به سوالات و درخواست‌های کاربران بهبود بخشد.

\section{ادغام Rocket.Chat}

ادغام Rocket.Chat با Rasa یک راه عالی برای افزودن قابلیت‌های چت‌بات AI به پلتفرم چت سازمانی شماست. با این ادغام، می‌توانید از Rasa برای ساخت چت‌بات‌هایی استفاده کنید که می‌توانند به سوالات و درخواست‌های کاربران پاسخ دهند، پشتیبانی مشتری ارائه دهند، یا حتی کارهایی مانند جمع‌آوری داده‌ها یا برنامه‌ریزی کارها را انجام دهند.

\lr{\markdownInput{README.md}}



\newpage